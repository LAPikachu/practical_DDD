\documentclass{article}

\usepackage{geometry}
 \geometry{
 a4paper,
 total={170mm,257mm},
 left=20mm,
 top=20mm,
 }
 
\usepackage{amsmath}
\usepackage{amsfonts}
\usepackage{amssymb}

\setlength\parindent{0pt} %No indent whole file

\title{How to : Install VTK CAMera for mm}
\date{3 June 2016}

\begin{document}
   \maketitle
   
\section*{Requisite:}
\begin{itemize}
\item 2.7 $\geq$ Python < 3.0 
\item VTK $\geq$ 6.0
\item Python VTK wrapper
\item f2py
\end{itemize}

  \tableofcontents

\section{(Optional) Desinstall older version of vtkam}

If you have install former vtkam version, the package has been installed by default to a similar path :
\\
\texttt{/opt/local/Library/Frameworks/Python.framework/Versions/2.7/lib/python2.7/site-packages/}
\\
If so, remove former vtkam package and its associate .egg from the last directory

\section{Check Python Path}

The Python path will allow python to find the vtkam package installation directory.
\\
Before installing vtkam, your PYTHONPATH should look something like :\\ \\
\texttt{echo \$ PYTHONPATH\\
:/VTK/Wrapping/Python/:/VTK/lib/:}

\section{Add vtkam package to Python Path}

Now, your PYTHONPATH should look something like :\\ \\
\texttt{echo \$ PYTHONPATH\\
:/VTK/Wrapping/Python/:/VTK/lib/:\\
/VTK/bin:/Users/Laurent/Documents/MicroMegas/SVN/src/outils/vtkam/install/lib/python}\\


$\cdots$ if the installation directory was called 'install'. It should be followed by '/lib/python'


\section{Install vtkam}

Go into the src directory of vtkam, usually 
\texttt{/src/outils/vtkam/}
then,\\ \\
\texttt{python setup.py install --home=/Your/Installation/Directory} (here: ./install)
\\ \\
\textbf{Note : the shell script 'install.sh' in \texttt{/src/outils/vtkam/} install vtkam in the \texttt{src/outils/vtkam/install} directory and add the path to the PYTHONPATH}


\section{Use vtkam}

You can use vtkam from any 1st subfolder of your mm directory (bin, in, out, Project,$\cdots$) by doing :

\texttt{import vtkam.vtkam} in your python environment.\\
or use \texttt{python vtkam\_start.py }in the \texttt{bin} folder


\end{document}